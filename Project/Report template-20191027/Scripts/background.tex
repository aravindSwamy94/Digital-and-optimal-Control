% ------------ shading ------------
%
%\pgfdeclarehorizontalshading
%	{horizontal}					% nome
%	{2cm}							% altezza della figura
%	{	rgb(0cm)=(1.0, 1.0, 1.0);	% colore iniziale
%		rgb(1cm)=(1.0, 1.0, 0.8)}	% colore finale
%%
%\pgfdeclareradialshading
%	{radial}						% nome
%	{\pgfpoint{1.0cm}{0.7cm}}		% posizione del centro di illuminazione (0,0 � in mezzo alla sfera)
%	{	rgb(0cm)=(0.9, 0.0, 0.0);	% colore iniziale
%		rgb(2cm)=(0.5, 0.0, 0.0)}	% colore finale
%
%\AddToShipoutPicture
%{
%	% ------------------ DECOMMENTARE LA PARTE CHE SERVE ------------------
%
%	\begin{tikzpicture}[remember picture,overlay,shading=horizontal]
%		%
%		% ---- shading lineare
%		% nota: aumentare o diminuire gli xshift ed yshift per posizionare il rettangolo
%		% correttamente all'interno del foglio che si sta usando
%		\node (aa)	[xshift=-\textwidth,yshift=-\textheight]	at	(current page.south west)	{};
%		\node (bb)	[xshift=+\textwidth,yshift=+\textheight]	at	(current page.north east)	{};
%		\shade[shading angle=-90]	(aa)	rectangle	(bb);
%		%
%		% ---- shading radiale
%		% TODO
%		%
%		% ---- scritta sullo sfondo
%		\node [rotate=-60,scale=10,text opacity=0.1] at (current page.center) {For peer review only};
%		%
%	\end{tikzpicture}
%
%	% ---- immagine
%	\put(0,0)
%	{
%		\parbox[b][\paperheight]{\paperwidth}
%		{
%			\vfill
%			\centering
%			\includegraphics[width=\paperwidth, height=\paperheight, keepaspectratio]{logo_unipd}
%			\vfill
%		}
%	}
%}
